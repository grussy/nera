\documentclass[12pt]{article}
\usepackage[paper=a4paper,left=30mm,right=30mm,top=35mm,bottom =35mm]{geometry}
\usepackage[utf8]{inputenc}
\usepackage[T1]{fontenc}
\usepackage{stmaryrd}
\usepackage{setspace}
\usepackage{mathrsfs}
\usepackage[ngerman]{babel}
\usepackage{amssymb}
\usepackage{amsmath}
\usepackage{fancyhdr}
\usepackage[dvips,unicode,colorlinks,linkcolor=black]{hyperref} 
\usepackage{graphicx}
\usepackage{float}

\begin{document}
\begin{titlepage}
\author{Grammel, Rolf } 
\title{Forstbenutzung - Technologie, Verwertung und Verwendung des Holzes} 
\maketitle
\newpage
\tableofcontents
\end{titlepage} 

\newpage
\section*{Forstbenutzung}
\begin{itemize}
\item Gewinnung, Ausformung und Verwertung von Holz
\item Technische Eigenschaften von Hölzern
\item Fällungs- und Ausnutzungsbetrieb
\item Holztransport
\item Abgabe und Verwertung von Holz 
\item Verwendung von Holz bei holzverbrauchendem Gewerbe
\item Nur mit Zusammenarbeit mit dem Holzverbraucher kann eine Maximierung des Holzwertes erreicht werden.
\item Produktionspolitik als Marketingmittel: Sorten- und Sortimentsbildung
\item Distributions- und Kontrahierungspolitik: Verkaufsverfahren
In Zusammenarbeit mit den Marktpartnern lassen sich Kosten im Forstbetrieb senken.
Z.B. Gewichtsverkauf von Indusrtielanghölzern
\end{itemize}




\section{Holztechnologie}


\subsection{Anatomie des Holzes}
\begin{itemize}
  \item Maßgeblich für die spezifischen Eigenschaften von Holz als Roh- und
        Werkstoff ist der mikro- und makroskopische Aufbau der chemischen
        Grundsubstanzen.
\end{itemize}


\subsubsection{Aufbau der Zellwand}
\begin{itemize}
  \item Das Grundgerüst der Zellwand ist Zellulose, aufgebaut als Glukosekette.
  \item Aufgebaut wird diese durch die Polymerisation von Glukose-Monomeren zu
        Zellulose-Makromolekülen 
        \begin{itemize}
          \item Ein Zellulose-Makromolekül besteht aus ca. 100-4000 (DP)
                Glukose-Monomeren
          \item DP = Durchschnittlicher Polimerisationsgrad
        \end{itemize}
   \item Mehrere Makromoleküle bilden dann Micelle und ca. 100 davon bilden
         Micellarstränge
         \begin{itemize}
           \item 10-20 Micellarstränge verbinden sich zu zu 
                 Zellulosemikrofibrillen welche dann somit aus 1000-2000
                 Makromolekülen bestehen.
           \item Die Zwischenräume zwischen den Micellarsträngen werden
                 Interfibrillar- oder Intermicellarräume genannt.
         \end{itemize}
    \item Die Zellulosemikrofibrillen sind in den verschiedenen Schichten der Zellwand
          unterschiedlich dicht und strukturiert gelagert:
          \begin{itemize}
            \item ML = Mittellamelle: strukturlose Grundsubstanz aus überwiegend Pektin
            \item PW = Primärwand: wenig gerichtetes lockeres Netz aus Zellulosefibrillen
            \item SW = Sekundärwand: dichtes Zellulosefibrillennetz, spiralig
                  die Längsachse der Fasern umlaufend
            \item TW = Tertiärwand: Fibrillen weniger dicht, meist parallel strukturiert.
          \end{itemize}
     \item Der Hohe DP der Zellulose ist die Grundlage hinsichtlich der
           Verwendungsmöglichkeit von Holz.
     \item Das Zellulose-Grundgerüst wird durch Kittsubstanzen, hauptsächlich Lignin
           verhärtet
           \begin{itemize}
             \item Verholzung: Zellulosefibrillen werden umhüllt, dichte der
                   Zellwand nimmt zu und die Elastizität ab. Besonders hart wird
                   diese Schicht in den Wandschichten mit lockerem
                   Zellullosegerüst.
           \end{itemize}
           
     \end{itemize}



\subsubsection{Verschiedene Zellarten als Grundbauelemente des Holzes}
\begin{itemize}
  \item Das Kambium trennt die Rinde vom innen liegenden Holz 
  \item Holz differenziert sich in verschiedene Zellarten
  \item Posenchymatische Zellen = faserige, spindelförmige Zellen
  \begin{itemize}
    \item Tracheiden: schlanke, spitz zulaufende Zellen, Hauptbauelement des
          Nadelholzes, dienen der Festigung und Wasserleitung, Wanddicke und
          Lumen 2-4$\mu$, im Durchmesser 20-40$\mu$
     \item Holzfasern (Fasertracheiden, Librifromfasern): dienen der Festigung
           im Laubholz, dickwandig und englumig, Länge 1-1,5 mm, im Durchmesser
           0,02-0,05mm.
      \item Tracheen: Durch Auflösung der Querwände der andeinanderstoßenden
            tonnenförmigen Zellen entstehen die Gefäße des Laubholzes. Länge
            der Gefäßglieder 0,1-0,5mm (Eiche Gefäßlänge bis 18m
            nachgewiesen). Im Durchmesser ringporige im Früholz über 0,4mm,
            zerstreutporige bis 0,1mm.
  \end{itemize}
  \item Parenchymatische Zellen: Backsteinförmig, großlumige speichernde Zellen
  \begin{itemize}
    \item Stangenparenchym, Strahlenparenchym
    \item Epithelzellen: Exkretzellen, die Harzkanäle des Nadelholzes umgeben
    \item Harzkanäle: Bildung durch Ausweitung von Interzellularräumen
  \end {itemize}
  \item Holzzellen haben 3 Aufgaben zu erfüllen: Wasserleitung, Festigung,
        Speicherung
  \begin{itemize}
    \item Verbunden zu Wasserleitgewebe, Festigungsgewebe und Speicherungsgewebe.
  \end{itemize}
  \item Heimische Holzarten beinhalten im Durschnitt:
  \begin{itemize}
    \item Nadelholz: 90-95\% Tracheiden, 4-10\% Parenchym, 1\% Harzkanäle (nicht
    Tanne/Eibe)
    \item Laubholz: 40-60\% Fasern/Tracheiden, 20-40\% Gefäße, 10-30\%
    Parenchym.
    \end{itemize} 
\end{itemize}
  

\subsubsection{Aufbau des Holzes aus den Bauelementen}
\begin{itemize}
  \item Aufbau des Holzes aus verschiedenen Zellarten unterscheidet sich
  arttypisch außerdem genetisch, standörtlich, jahreszeitlich und altersmäßig.
  \item Nadelholz: Gleichmäßiger Aufbau
  \begin{itemize}
    \item Grundgewebe aus Längstracheiden in radialer Reihung
    \item Frühholz weitlumig, Spätholz englumig
    \item In geringem Umfang Markstrahltracheiden zur Wasserleitung in radialer Richtung
    \item Parenchymgewebe meist als einreihige Markstrahlen ausgebildet
    \item Bei den meisten Holzarten Harzkanäle als schizogene Bildung
  \end{itemize}
  \item Laubholz: differenzierter im Aufbau
  \begin{itemize}
    \item Zertreutporig: ungleichmäßige Verteilung der Gefäße im Jahrring,
    kein wesentlicher Unterschied in den Gefäßdurchmessern. Bu, Wb, Ah, Bi, Er, Pa, We
    \item Ringporig: Früholzgefäße in 2-3 tangetialen Reihen im Durchmesser
    größer als Spätholzgefäße. Ei, Ul, Es, Ka, Ro
    \item Halbringporig: Früholzgefäße in einer tangentialen Reihe,
    Durchmesser kaum größer als Spätholzgefäße Li, Nussbäume
  \end{itemize}
  \item Jahrringbildung: Jahreszeitlicher Rhythmus der Holzbildung besonders
  stark bei Nadelholz oder ringporigem Laubholz. Tropische Hölzer besitzen
  keine Jahrringe (Zuwachszonen). Bedeutung bei der Zuwachs- und Ertragskunde
  (Qualitätsmerkmal): Ei, Ki, DG, Fi
  \item Verkernung: Mit zunehmendem Alter scheiden Tracheiden  aus der
  Wasserleitung aus, der Wasserstrom wird unterbrochen und die Zellöffnungen
  durch verschiedene Vorgänge verschlossen. (Thyllenbildung, Tüpfelverschluss)
  In die Zellhohlräume und Zellwände werden meist auch Kernstoffe wie Harze,
  Gerbstoffe und Farbstoffe eingelagert. Bei ringporigen Hölzern transporiert
  der äußerste Jahrring die Hauptmenge Wasser, bei Nadelbäumen die äußersten 5
  Ringe, während zerstreutporige Bäume wesentlich mehr Ringe an der
  Wasserleitung beteiligt sind.
  \begin{itemize}
    \item Kernholz:Sind die veränderten Zonen des Holzes die kein Wasser
    mehr transportieren.
    \begin{itemize}
      \item In der Regel trocker, schwerer und widerstandsfähiger gegen
      Pilzbefall als Splintholz.
      \item Kernhölzer: besitzen ausgeprägt gefärbten Kern (Ki, Lä, Dg, Es, Ul,
      Ei, Ka, Ro)
      \item Reifhölzer: trockenes Kernholz das sich farblich nicht vom
      Splintholz unterscheidet (Fi, Ta)
      \item Die Farbe des Kernholzes hat meist entscheidenden Einfluss auf den
      Holzwert. Insbesondere bei BU verhindert der Verschluss der Zellen eine
      ausreichende Imprägnierung.
    \end{itemize}
  \end{itemize}
  \begin{itemize}
    \item Splintholz: Sind die jüngeren noch leitungsfähigen Zonen des Holzes 
    \begin{itemize}
      \item Splinthölzer: Meist zerstreutporige Hölzer, bei denen sich der
      Feuchtigkeitsgehalt zwischen  Splint und Kern sich kaum unterscheidet (Wb, Ah, Bi)
      \end{itemize}
  \end{itemize}
  \item Reaktionsholz:
  \begin{itemize}
    \item Druckholz: Nadelhölzer (ligninreich
    \item Zugholz: Laubhölzer (Zellulosereich)
    \end{itemize}
\end{itemize}
 
\subsubsection{Juveniles und adultes Holz}
\begin{itemize}
  \item Juveniles Holz: sind die markumgebenden inneren Jahrringe
  \begin{itemize}
    \item Jugendholz ist deutlich leichter als älteres
  \end{itemize}
  \item Adultes Holz: Äußeren Jahrringe (normales Holz)
  \item Der Übergang zwischen adultem und juvenilen Holz ist fließend und
  nicht klar abzugrenzen
\end{itemize}





\subsection{Chemie des Holzes}
\subsubsection{Elementarzusammensetzung}
\begin{itemize}
  \item Die Grundsubstanzen aller Holzarten sind ziemlich gleich 
  \begin{itemize}
    \item Kohlenwasserstoffe (50\% C, 43\% O, 6\% H, 1\% N, 1\% Kationen)
  \end{itemize}
  \item Veränderung der Holzeigenschaften mit dem Holzalter:
  \begin{itemize}
    \item Vom Mark zur Rinde steigt die Faserlänge, Zellwandstärke, 
    Dichte, Spätholzanteil, Festigkeit, Tangentialschwindung, 
    Astigkeit.
    \item Vom Mark zur Rinde sinkt der Mikrofibrillenwinkel, 
    Drehwüchsigkeit, Longitudinalschwindung und die Holzfeuchte
    \end{itemize}
\end{itemize}



\subsubsection{Gerüstsubstanzen}
\begin{itemize}
  \item Zellulose: Makromolekül aus Glukosebausteinen (1000-4000 DP) => 45\%
  \item Holzpolyosen: Hemizellulosen (Pektin, Pentosane etc.) (100-150 DP) =>
  25\%
  \item Lignin: Amorphe Makromoleküle (wenig strukturierter Füllstoff) => 25\%
  \item Zellstoff: Überwiegend Zellulose sonst Polyosen, Ligninreste (500 DP)

\end{itemize}




\subsubsection{Akzessorische Bestandteile}
\begin{itemize}
  \item Öle, Fette, Harze, Stärken, Gerb- und Farbstoffe, Gift- und
  Mineralstoffe => 5\%
\end{itemize}


\subsubsection{Chemische Zusammensetzung von Normalholz und Richtgewebe
(Reaktionsholz)}

\subsection{Physik des Holzes}
\subsubsection{Dichte des Holzes}
\begin{itemize}
  \item Die Holzdichte ist ein zentraler Begriff und hängt eng  zusammen mit:
  \begin{itemize}
    \item den Festigkeitseigenschaften von Holz 
    \item Brennwert des Holzes
    \item Ausbeute in der industrieholzverarbeitenden Industrie
    \item Holzernte und Transportkosten
  \end{itemize}
  \item DIN 1306: Dichte = Masse/ Volumen
  \item Reindichte: Volumen des Feststoffes allein
  \begin{itemize}
    \item Dichte der Zellwandsubstanzen ziemlich einheitlich 1,5 $\frac{g}{cm^3}$ (1,6
    Zellulose, 1,4 Lignin)
  \end{itemize}
  \item Rohdichte: Volumen der ganzen Stoffmenge einschließlich der
  Zwischenräume
  \begin{itemize}
    \item Dichte des Holzes bei Feuchtigkeitsgehalt u\%
  \end{itemize}
  \item Darrdichte: Dichte des absolut trockenen Holzes
  \begin{itemize}
    \item Abhängig vom Verhältnis Zellwandsubstanz zu Porenvolumen
  \end{itemize}
  \item Raumdichte: Gewicht pro Frischvolumen (Theoretische Größe)
  \begin{itemize}
    \item Umrechnungsmaß für vom Verkaufsmaß für Industrieholz
    \end{itemize}
  \end{itemize}
 

\subsubsection{Bestimmung der Dichte}
\begin{itemize}
  \item Volumenermittlungsverfahren mit exakt ausgeformten und abgewogenen
  Probekörpern oder Bohrspänen:
  \begin{itemize}
    \item Stereometrisch: Ausmessen der Proben mittels Schieblehre/
    Bügelmessschraube
    \begin{itemize}
      \item Fehlerquellen Probekörper: durch die Rauhigkeit wird das Volumen
      vergrößert, Vertiefungen werden nicht erfasst
      \item Fehlerquellen Bohrspäne: Nach Trocknung ellipsenförmige
      Querschnitte und Rauhigkeit
    \end{itemize}
  \end{itemize}
  \begin{itemize}
    \item Xylometrisch: Messung der Wasserverdrängung mit Volumeter
    (Präparation Holz gegen Quellung)
    \begin{itemize}
      \item Fehlerquellen: Luftblasenbildung beim eintauchen und Wasserverlust
      bei herausnahme, die aber durch herabsetzen der Grenzflächenspannung
      (Netzmittel) behoben werden können.
    \end{itemize}
  \end{itemize}
  \begin{itemize}
    \item Hydrostatisch: Auftrieb  eines Körpers = Gewicht
    der von ihm verdrängten Flüssigkeitsmenge V = G – G' ( G = Absolutes
    Gewicht, G' = Gewicht d. Flüssigkeit)
    \begin{itemize}
      \item Fehlerquellen: Nimmt Flüssigkeit auf und verändert Gewicht und
      Volumen die aber durch wasserabweisenden Überzug oder z.B. Petroleum
      behoben werden können.
    \end{itemize}
  \end{itemize}
  \begin{itemize}
    \item Strahlungsmessung: Lichtdurchlässigkeit an Holzmikroschnitten
    \begin{itemize}
      \item Nur innerhalb einer Holzart, da gleiche Dichte bei
      unterschiedlichen Arten unterschiedliche Lichtdurchlässigkeit besitzen.
      \item Röntgenstrahlung: Absorbtions- und Reflexionsmessung bei bekannter
      Holzstärke und -feuchte
      \item Mikrowellen (polarisiert): Messung der Faserverschiebung
    \end{itemize}
  \end{itemize}
  
\end{itemize}
 

     
  

\subsubsection{Mittelwerte und Variabilität der Dichte}


\paragraph{Meßgrößen}
\paragraph{Quellung und Schwindung}
\paragraph{Anisotropie}
\paragraph{Geweichtsvermessung von Industrieholz}
\subsection{Mechanische Eigenschaften des Holzes}
\subsubsection{Elastische Eigenschaften}
\subsubsection{Festigkeitseigenschaften}
\paragraph{Statische Festigkeit}
\paragraph{Dynamische Festigkeit - Bruchschlagbarkeit}
\paragraph{Dauerfestigkeit}
\subsubsection{Härte und Abnutzungswiderstand}
\subsubsection{Prüfung des Holzes}
\subsubsection{Zerstörungsfreie Prüfung}
\subsubsection{Einfluss verschiedener Faktoren auf die Festigkeitseigenschaften}
\subsubsection{Thermische Eigenschaften des Holzes}
\subsubsection{Elektrische Eigenschaften des Holzes}
\subsubsection{Akkustische Eigenschaften}
\subsection{Rundholzfehler}
\section{Holzschutz}
\subsection{Aufgaben}
\subsection{Natürliche Dauerhaftigkeit des Holzes}
\subsection{Organisatorischer und technischer Holzschutz}
\subsection{Baulich-konstruktiver Holzschutz}
\subsection{Chemischer Holzschutz}
\subsubsection{Typen von Holzschutzmitteln}
\subsubsection{Feuerschutz}
\subsubsection{Holzschutzverfahren}
\subsubsection{Problematik des chemischen Holzschutzes}
\subsection{Biologischer Holzschutz}
\subsubsection{Umweltverträglichkeitsprüfung - UVP}
\subsubsection{RAL-Gütezeichen}
\subsubsection{Heißluftverfahren}
\subsubsection{Schädlingsbekämpfung bei befallenem Holz}
\section{Rohholzverwertung}
\subsection{Einführung}
\subsubsection{Instrumente des Marketings}
\subsection{Planung des Holzeinschlags}
\subsubsection{Zielsetzungskomponente der Holzeinschlagsplanung}
\paragraph{Wirtschaftliche Zielsetzungskomponenten}
\paragraph{Waldbauliche Zielsetzungskomponenten}
\paragraph{Sonstige Zielsetzungskomponenten}
\subsubsection{Planungsdaten und mögliche Beeinflussung durch den Betrieb}
\paragraph{Forstliche Planungsdaten}
\paragraph{Erfordernisse Holzmarkt}
\subsubsection{Darstellung des Ablaufs der Holzeinschlagsplanung}
\subsubsection{Ausgewählte Holzmarktdaten}
\subsection{Vermessung und Sortierung des Rundholzes}
\subsubsection{Ziele der Sortenbildung}
\subsubsection{Prinzipielle Kriterien der Holzsortierung}
\subsubsection{Holzvermessung als Voraussetzung der Sortenbildung}
\subsubsection{Geschichtliche Entwicklung der Holzvermessung und Sortierung}
\subsubsection{Die Bestimmungen der Forst-HKS}
\paragraph{Grundlagen}
\paragraph{Gesetzliche Handelsklassen für Rohholz}
\paragraph{Beurteilung und Kritik}
\subsection{Verkauf des Rohholzes}
\subsubsection{Verkaufsvorbereitung}
\subsubsection{Verkaufsarten}
\subsubsection{Verkaufsverfahren}
\subsubsection{Allgemeine Verkaufs- und Zahlungsbedingungen}
\subsection{Holzverkauf ins Ausland}
\section{Holzverwendung}
\subsection{Einführung}
\subsubsection{Verwendungsmöglichkeiten des Holzes}
\subsection{Verwendung des Holzes in runder oder wenig veränderter Form}
\subsection{Verwendung des Holzes nach mechanischer Verformung unter Beibehaltung des natürlichen Gefüges - Schnittholz}
\subsubsection{Struktur der Sägeindustrie}
\subsubsection{Die wichtigsten Einrichtungen und Arbeitsgänge im Sägewerk}
\paragraph{Rundholzplatz}
\paragraph{Sägehalle}
\paragraph{Schnittholzoptimierung}
\paragraph{Schnittholzplatz}
\subsubsection{Vermessen und Sortieren von Schnittholz}
\subsubsection{Erzeugnisse der Sägeindustrie}
\subsubsection{Schnittholzverwendung}
\subsubsection{Arbeitsmedizinische Beurteilung von Holzstaub}
\subsection{Furniere}
\subsubsection{Technik der Furnierherstellung}
\subsection{Holzwerkstoffe-Verwendung von Holz nach weitestgehender
mechanischer Verformung und Aufbau eines neuen Gefüges}
\subsubsection{Umwandlung von Holz in Holzwerkstoffe}
\subsection{Verwendung des Holzes nach Zerlegung in Fasern und Aufbau eines
neuen Gefüges (Holzschliff, Zellstoff, Papier)}
\begin{itemize}
  \item Fasergefüge des Holzes wird durch industriellen Aufbau zerstört
  \begin{itemize}
    \item Faserrohstoff (aus möglichst langen Fasern) für die Weiterverarbeitung
    zu Faserplatten, Papier, Chemierohstoff
    \item Nadelholz, ist besonders gut geeignet wegen der langen Tracheichen
  \end{itemize}
  \item Verfahrensweisen der Zerlegung
  \begin{itemize}
    \item Mechanischer Aufschluss des Holzschliffs
    \item Chemischer Aufschluß bei Zellstoff
    \end{itemize}
\end{itemize}
\subsubsection{Holzschlifferzeugung}
\begin{itemize}
  \item Zerfaserung mittels Schleifsteine aus Rundholz, Hackschnitzeln etc.
  \item Weißschliff:
  \begin{itemize}
    \item Fi, Ta, Pa keine Vorbehandlung des Holzes (90-95\% Ausbeute)
  \end{itemize}
  \item Braunschliff:
  \begin{itemize}
    \item Harzreiche Hölzer (Ki) mit Vorbehandlung des Holzes unter Druck und
    Wasser (80-85\% Ausbeute)
  \end{itemize}
  \item Chemischer Schliff:
  \begin{itemize}
    \item Vorbehandlung Holz mit Chemikalien zur Ligninauflockerung
  \end{itemize}
  \item Faserstoff:
  \begin{itemize}
    \item Faserlangstoff (> 1mm) verbessern die dynamischen Eigenschaften
    \item Faserkurzstoff (0,5-1mm) Verbesserung statischer Eigenschaften
  \end{itemize}
  \item Feinstoff:
  \begin{itemize}
    \item Schleimstoff verbessert statische Eigenschaften und verringert die
    Entwässerungsgeschwindigkeit
    \item Mehl-und Staub Stoff verschlechtern alle mechanischen Eigenschaften
    \item Dichtes härteres, glattes Papier 
  \end{itemize}
  \item Die Stoffe werden Aufbereitet (sortiert), gebleicht und eingedickt
\end{itemize}

\subsubsection{Zellstoffherstellung}
\begin{itemize}
  \item Zerleglung des Holzes in Fasergefüge und die Entfernung von Lignin
  durch chemische Behandlung, hohen Druck und Temperatur ergibt Zellstoff
  \item Zellstoff: Cellulose, Hemicellulose, Rest- Ligninen
  \item Cellulose: Hauptbestandteil der Zellwand
  \item Papierzellstoff: Hochwertige Druck- und Schreibpapiere (Fi, Ta, Ki)
  \item Kunstfaser Zellstoff: Weiterverarbeitung zu halb synthetischen Fasern
  (Viskose)
  \item Nutzung überwiegend Sulfit- und Sulfatverfahren
\end{itemize}

\subsubsection{Papierherstellung}
\begin{itemize}
  \item Papier ist 3D Netzwerk von Fasern in das Faserfragmente, Füll- und
  Farbstoffe eingelagert sind
  \begin{itemize}
    \item Halbstoff  + Zuschlagsstoff = Ganzstoff
  \end{itemize}
  \item Technik der Papierherstellung:
  \begin{enumerate}
    \item Gewinnung der Halbstoffe (Schliff, Zellstoff) 
    \item Herstellung des Ganzstoffes aus Halbstoffen und Zusatzmaterial
    \item Bildung des PapierblätterBildung des Papierblätter
    \item Vollendungsarbeiten am Papier
  \end{enumerate}
  \item Aufbereitung Altpapier:
  \begin{itemize}
    \item Auflösung durch Stofflöser und Beseitigung Verunreinigungen
    \item Deinking, abtrennen Kunstoffbeschichtung, Faserfraktionierung
    \item Bei Recycling werden die Fasern verkürzt (Zufuhr Holzschliff)
  \end{itemize} 
\end{itemize}

\subsubsection{Kunststoffe und Cellulosische Fasern}
\begin{itemize}
  \item Rohstoffe sind Zellstoffe aus Laub- und Nadelholz
  \begin{itemize}
    \item Viskoseseide, Vulkanfiber, Zellglas, Zelluloid
  \end{itemize}
\end{itemize}

\subsection{Holz zur Energieerzeugung und als Chemierohstoff}
\subsubsection{Energieerzeugung aus Holz weltweit}
\begin{itemize}
  \item In Entwicklungsländern meist hauptsächliche Energiequelle
  \item Industrielle Länder überwiegend fossile Energieträger
\end{itemize}

\subsubsection{Verfahren zur Umwandlung von Holzbiomasse in Energie}
\begin{itemize}
  \item Verbrennung, (Heizwert): wird von Art und Wassergehalt bestimmt
  \item Pyrolyse: Holzverkohlung, Holzvergasung, Holzverflüssigung
  \item Holzhydrolyse
\end{itemize}
\subsection{Nebennutzung}
\begin{itemize}
  \item Gewinnung von Produkten die keinen primären Holzcharakter haben
  \begin{itemize}
    \item Weihnachtsbaum und Schmuckreisig
    \item Kies- und Steinabbau
    \item Rindenverwertung
  \end{itemize}
\end{itemize}

\subsubsection{Nebenprodukte der Waldbäume}
\begin{itemize}
  \item Rinde: Verarbeitung mit Holz, Zellstoff, Schlifferzeugung,
  Industriepellets, Rindenmulch, Plattenwerkstoffe, Rindenhumus,
  Rindenkultursubstrate, Gerbstoffgewinnung, Korknutzung
  \item Harze, Terpentene, Tallöl
  \begin{itemize}
    \item Lebendharzung, Extraktharzung (Seifen, Lacke, Weichmacher, Leime)
    \end{itemize}
  \item Früchte: Mast, Saatgut
  \item Futterstoffe: Zweig- und Laubfutter
  \item Gummi-Kautschuk
  \item Öle, Fette
  \item Zucker, Farben, Pilze, Weihnachtsbaum, Zierreisig, Bienenweide
\end{itemize}

\subsubsection{Nebenprodukte des Waldbodens}
\begin{itemize}
  \item Gras, Kräuter, Moos: Weide für Haus- und Wildtiere
  \item Landwirtschaftlicher Zwischenanbau
  \item Streunutzung
  \item Beeren, Früchte, Bodenpilze
  \item Knollen, Wurzeln, Heil- und Gewürzpflanzen
  \item Topfnutzung
  \item Kies, Sand, Steine, Erde
\end{itemize}


\end{document}
