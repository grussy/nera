\documentclass[12pt]{article}
\usepackage[paper=a4paper,left=30mm,right=30mm,top=35mm,bottom=35mm]{geometry}
\usepackage[utf8]{inputenc}
\usepackage[T1]{fontenc}
\usepackage{stmaryrd}
\usepackage{setspace}
\usepackage{mathrsfs}
\usepackage[ngerman]{babel}
\usepackage{amssymb}
\usepackage{amsmath}
\usepackage{fancyhdr}
\usepackage[dvips,unicode,colorlinks,linkcolor=black]{hyperref} 
\usepackage{graphicx}
\usepackage{float}

\pagestyle{fancy}
\lfoot{}
\rfoot{}
\cfoot{\thepage}
\fancyhead[L]{}
\renewcommand{\headrulewidth}{0.6pt}
\renewcommand{\footrulewidth}{0.6pt}
\setlength{\headheight}{16pt}
\setlength{\parindent}{0pt}
% Für die Wahl der Schriftart
\newcommand{\changefont}[3]{
\fontfamily{#1} \fontseries{#2} \fontshape{#3} \selectfont}

%\renewcommand{\vec}[1]{\mathbf{#1}}
\renewcommand*\vec[1]{{\mbox{\boldmath\ensuremath{#1}}}}

\begin{document}
% keine Hurenkinder und Schusterjungen
\clubpenalty = 10000
\widowpenalty = 10000 
\displaywidowpenalty = 10000

\onehalfspacing
% Schriftart
\changefont{ptm}{m}{n} 

\begin{titlepage}
\author{}
\title{Zusammenfassung: A Tutorial on Graph-Based SLAM}
\date{\today} 
\maketitle
\thispagestyle{empty}
\end{titlepage}


\tableofcontents
\thispagestyle{empty}
\newpage
\pagenumbering{arabic}

\section{Introduction}
SLAM (simultaneous localization and mapping) bietet eine Alternative zur Verwendung von Systemen wie GPS, vor allem wenn solche Systeme nicht
verfügbar sind. The bestehenden Ansätze lassen sich in die zwei Klassen \textit{filtering} und \textit{smoothing} unterteilen.
Ein intuitiver Weg solch ein SLAM Problem anzugehen ist die sogenannte ``graph-based formulation''. Hierzu wird ein Graph erstellt, dessen
Stützstellen Posen (also Ort und Raumwinkel des Roboters zu bestimmten Zeiten) darstellen und die Verbindungen der Stützstellen durch Sensordaten
aufgestellte Zwangsbedingungen beschreiben. Diese Zwangsbedingungen sind nicht zwingend eindeutig da die Messdaten von einem Rauschen überlagert
sind.

Ist ein solcher Graph gefunden, gilt es die Konfiguration der Stützstellen zu finden die am besten zu den durch die Messungen gegebenen
Zwangsbedingungen passt. Die hierfür benötigten Minimierungslösungen sind sehr Komplex, weshalb diese Methode einige Zeit gebraucht hat sich zu 
etablieren.

\subsection{Filtering}
\textit{Filtering} ist eine live Zustandsberechnung, wobei der Zustand aus der aktuellen Roboterposition und der Karte besteht. Die Berechnung
wird durch das Aufnehmen weiterer Daten immer weiter verbessert. Die Filtering Methoden werden als on-line SLAM Methoden bezeichnet um ihre
``wachsende'' Natur zu unterstreichen.

\subsection{Smoothing}
Die \textit{smoothing} Ansätze verwenden im Gegensatz zu den \textit{filtering} Ansätzen den Kompletten Datensatz an Karten und 
Positionsinformationen. Diese Ansätze behandeln das sogenannte ``full SLAM problem''. Typischerweise werden hier Minimierungstechniken wie die
Methode der kleinsten Quadrate verwendet.

\section{Probabilistic formulation of SLAM}
Die Lösung des SLAM Problems besteht darin, die Robotertrajektorie und die Karte der Umgebung zu berechnen während der Roboter sich in ihr bewegt.
Wegen dem starken Rauschen der Sensoren wird das SLAM Problem meistens mit wahrscheinlichkeitstheoretischen Mitteln behandelt. Man nimmt an der
Roboter bewegt sich entlang einer Trajektorie welche durch eine Reihe von zufälligen Variablen beschrieben wird
$\vec x_{1:T} = \left\{\vec x_1,...,\vec x_T\right\}$. Während dieser Bewegung nimmt der Roboter eine Reihe von odometrischen Daten
$\vec u_{1:T} = \left\{\vec u_1,...,\vec u_T\right\}$ und Bilder der Umgebung $\vec z_{1:T} = \left\{\vec z_1,...,\vec z_T\right\}$.
\\

Die Lösung des ``full SLAM'' Problems besteht nun darin die bedingte Wahrscheinlichkeit der Robotertrajektorie $\vec x_{1:T}$ und Karte $\vec m$ aus
gegebenen Messdaten und einer Anfangsposition $x_o$ zu Berechnen:
\begin{align}
 \label{posterior}
 p\left(\vec x_{1:T},\vec m|\vec z_{1:T}, \vec u_{1:T}, \vec x_0\right)
\end{align}
Die Anfangsposition $\vec x_0$ definiert die Position der Karte und ist beliebig wählbar. Die Roboterposen und odometrischen Daten werden 
generell als 2D oder 3D Transformationen in den Euklidischen Gruppe SE(2) bzw. SE(3) dargestellt. Für Darstellung der Karte bestehen diverse
Möglichkeiten. Die Wahl der Darstellung hängt von Faktoren wie den Benutzen Sensoren, der Art der Umgebung oder dem Berechnungsalgorithmus ab.
Unabhängig von der Wahl der Darstellung definiert sich die Karte durch die Messungen und die Orte an denen die Messungen erlangt wurden.
\\

Da die Berechnung der bedingten Wahrscheinlichkeit (\ref{posterior}) Arbeiten in hochdimensionalen Zustandsräumen umfasst, ist diese nur durch
die wohldefinierte Struktur des SLAM Problems zu lösen. Diese Struktur basiert auf der Annahme einer fixen Welt, und der Markov
Annahme\footnote{Die Eigenschaften jedes Zustands werden nur von den Eigenschaften des unmittelbar vorausgegangenen Zustands beeinflusst.}.

\subsection{Dynamic Bayesian network (DBN)}
Die Struktur des SLAM Problems kann mit einem DBN, einem grafischen Modell, welches einen stochastischen Prozess mittels eines gerichteten Graphen
beschreibt, behandelt werden. Der Graph enthält Knoten welche $\vec x_{1:T}$, $\vec u_{1:T}$, $\vec z_{1:T}$ und $\vec m$ repräsentieren. Die
Verbindungen der Knoten folgen einem periodischen Prinzip, welches von dem Zustandsübergangsmodell und dem Beobachtungsmodell charakterisiert wird.
Das SLAM Problem als DBN darzustellen betont seine zeitabhängige Struktur und zeigt, dass dieser Formalismus hervorragend geeignet ist um 
\textit{filtering} Prozesse zu beschreiben die verwendet werden können um SLAM Probleme anzugehen.

\subsubsection{Zustandsübergangsmodell}
Das Zustandsübergangsmodell $p\left(\vec x_t |\vec x_{t-1}, \vec u_t\right)$ wird von zwei Linien repräsentiert welche zu $\vec x_t$ führen und
stellt die Wahrscheinlichkeit, dass der Roboter sich zur Zeit $t$ in $\vec x_t$ befindet dar. Vorausgesetzt der Roboter hat sich zur Zeit $t-1$ am
Ort $\vec x_{t-1}$ befunden und unterwegs die odometrische Messung $\vec u_t$ erlangt.

\subsubsection{Beobachtungsmodell}
Das Beobachtungsmodell $p\left(\vec z_t|\vec x_t, m_t\right)$ beschreibt die Wahrscheinlichkeit, dass der Roboter die Beobachtung $\vec z_t$
macht, sofern er sich am Ort $\vec x_t$ in der Karte befindet. Dargestellt wird das Modell Durch Pfeile die auf $\vec z_t$ zeigen. Die Beobachtung
$\vec z_t$ hängt nur von der gegenwärtigen Postion $\vec x_t$ und der statischen Karte $\vec m$ ab.

\subsection{Graph-based SLAM}
Eine Alternative zum DBN bietet die \textit{Graph-based} Formulierung des SLAM Problems. Hier werden die Posen des Roboters durch Knoten im Graphen
dargestellt und mit der entsprechenden Position in der Umgebung beschriftet. Räumliche Zwangsbedingungen zwischen den Posen welche aus Beobachtungen
$\vec z_t$ oder odometrischen Messungen $\vec u_t$ resultieren werden durch Verbindungslinien zwischen den Knoten im Graphen dargestellt. Eine solche
Zwangsbedingung besteht aus der Wahrscheinlichkeitsverteilung über der relativen Transformation zwischen den zwei Posen. Ist ein solcher Graph konstruiert
so wird versucht die Konstellation von Posen zu finden unter der die Zwangsbedingungen zwischen den Posen am besten erfüllt sind.

\subsubsection{Frontend}
Die Erstellung des Graphen bezeichnet man üblicherweise als \textit{frontend} oder \textit{graph construction}. Der Graph wird aus den rohen Messdaten
der Sensoren erzeugt.
Diese kommen entweder von odometrischen Messungen oder durch ausfluchten bzw. abstimmen der Beobachtungen welche an den zwei Punkten gemacht wurden.

\subsubsection{Backend}
Die Optimierung des Graphen wird als \textit{backend} oder \textit{graph optimization} bezeichnet. Die Optimierung hängt nicht vom verwendeten Sensor
ab, sonder nur von der abstrakten Darstellung der Daten.

\section{Graph-based SLAM}
Die Messdaten werden durch die Verbindungen im Graph dargestellt und können als ``virtuelle Messungen'' angesehen werden. Genauer repräsentiert die
Verbindung zweier Posen eine Wahrscheinlichkeitsverteilung über den relativen Orten der Posen im Bezug auf ihre gegenseitige Messungen. Allgemein ist
das Beobachtungsmodell $p\left(\vec z_t|\vec x_t, m_t\right)$ moltimodal weswegen die Annahme einer Gaussverteilung nicht zutrifft. Dies bedeutet, 
dass eine einzelne Beobachtung $\vec z_t$ mehrere potentielle Verbindungen zwischen verschiedenen Posen darstellen könnte. Deshalb müssen die Gesamtheit 
der Verbindungen im Graph selbst als Wahrscheinlichkeitsverteilung beschrieben werden. Eine direkte Behandlung der Multimodalität würde zu einer 
kombinatorischen Explosion der Komplexität führen. Aus diesem Grund schätzt man meistens die Wahrscheinlichste Topologie ab, man muss also die 
Zwangsbedingung bestimmen die am wahrscheinlichsten aus einer Beobachtung resultiert. Diese Entscheidung basiert auf der Wahrscheinlichkeitsverteilung über
den Roboterposen. Bekannt ist dieses Problem unter dem Namen \textit{data association} und wird normalerweise vom Frontend durchgeführt. Um die richtige
\textit{data association} zu berechnen muss dem Frontend eine konsistente Abschätzung der unbedingten Wahrscheinlichkeit über der Robotertrajektorie
$p\left(\vec x_{1:T}|\vec z_{1:T}, \vec u_{1:T}\right)$ gegeben sein. Dies erfordert eine Verschachtelung der Ausführung von Frontend und Backend während
der Roboter die Umgebung erkundet. Weswegen die Genauigkeit und Geschwindigkeit des Backends für eine gutes SLAM System ausschlaggebend ist.
\\

Sind die Beobachtungen von Gaußverteiltem Rauschen überlagert und die \textit{data association} bekannt, ist das Ziel des Algorithmus die Anpassung einer
Gaußverteilung an die bedingte Wahrscheinlichkeitsverteilung der Robotertrajektorie. Dies beinhaltet die Berechnung des Erwartungswerts dieser 
Gaussverteilung als Konfiguration der Knoten die die Wahrscheinlichkeit der Beobachtungen maximiert.


\end{document}
