\documentclass[12pt]{article}
\usepackage[utf8x]{inputenc}
\usepackage[paper=a4paper,left=30mm,right=30mm,top=35mm,bottom =35mm]{geometry}
\usepackage{stmaryrd}
\usepackage{setspace}
\usepackage{mathrsfs}
\usepackage[ngerman]{babel}
\usepackage{amssymb}
\usepackage{amsmath}
\usepackage{fancyhdr}
\usepackage[dvips,unicode,colorlinks,linkcolor=black]{hyperref} 
\usepackage{graphicx}
\usepackage{float}


\usepackage[multiple,marginal,ragged,hang]{footmisc}
\usepackage{cite} 
\usepackage{hyperref}  
\usepackage[babel,german=quotes]{csquotes}
\usepackage[T1]{fontenc}
\usepackage{multibib}


\newcommand{\citefooturl}[1]{\footnote{\url{#1}}}
\newcommand{\citefooturldate}[2]{\footnote{\url{#1} [#2]}}

%opening
\begin{document}

\clubpenalty = 10000
\widowpenalty = 10000 
\displaywidowpenalty = 10000

\onehalfspacing 


\begin{titlepage}

\title{Multifunktionale Waldwirtschaft in Deutschland - Leerformel oder erreichbares Ziel der Forstpolitik?}
\author{\\ \\ \\ \\ vorgelegt von \\ \\ Hannah Sharaf \\ Matr.-Nr. 2304883}
\date{am \\ 10.01.2011}
\maketitle
\vfill {\noindent Institut für Forst- und Umweltpolitik an der Albert-Ludwigs-Universität Freiburg. \\ Verantwortlicher Dozent: Herr Prof. Dr. Uwe Schmidt.}
\thispagestyle{empty}
\newpage

\end{titlepage} 


\section*{Einführung}
Mit offenen Augen durch den Mooswald wandeln, bedeutet ein Stück Geschichte zu betrachten.
Auch wenn für viele Menschen der Stadtwald die unberührte Natur darstellt, sind überall Spuren
längst vergangener Zeit zu finden. Kaum einer kann sich vorstellen, dass dieses gerne zur
Erholung genutzte Gebiet, einmal eine menschenfeindliche Sumpflandschaft war.

Der ehemals dichte Urwald aus Erlen, Eschen, Weiden, Pappeln aber auch Tannen wurde bereits von den
Römern genutzt, weshalb die Tanne aus diesen Wäldern verschwand. Schon damals wirkte der Mensch
sich auf die Gestalt und Entwicklung des Waldes aus. Deshalb können wir seit dem nicht mehr von
einem Urwald sprechen.

Erstmals urkundlich erwähnt wurde der Mooswald im Jahre 1008. König Heinrich II. übertrug dem Basler
Bischhof Adalbero die Jagdrechte (Wildbann) über die Waldfläche. Spätestens seit 1289 ist das Waldgebiet
im Besitz der Stadt Freiburg. Kahlschlag, landwirtschaftliche Nutzung, Wiederaufforstungen, Entwässerung, 
Kiesabbau haben den Wald so geprägt, dass keine Naturlandschaft mehr zu betrachten ist, sondern eine vom 
Menschen geschaffene Kulturlandschaft.\footnote{Amtsblatt Juni 2008}

Heute steht der Freiburger Mooswald zu weiten Teilen unter Schutz. Die gesamte Fläche gehört dem
Schutzgebietsnetz Natura 2000 an, ob als Vogelschutz- oder FFH-Gebiet. Zudem sind große Bereiche
als Landschaftsschutz- und Naturschutzgebiete ausgewiesen. Auch einen Bannwald und andere kleinflächigere
Schutzkategorien wie z.B. Denkmalschutz sind zu finden.\citefooturldate{http://www.badische-zeitung.de/freiburg/schutzgebiet-mooswald--23461641.html}{30.11.10} 
Als stadtnaher Wald, steht dem Mooswald, neben dem
großflächigen Schutz auch ein hoher Besucherstrom von Erholungssuchenden entgegen. Zu guter letzt, darf man
auch wirtschaftliche Nutzung des Waldes als Holzlieferant nicht vergessen, die ebenfalls auf den gleichen
Flächen stattfindet.

Daraus ergeben sich natürlich einige Fragen:\\ 
Kann eine Nutzung des Waldes auch Schutz und Erholung gewährleisten? \\
Ist der Einfluss des Menschen auf die Natur immer etwas schlechtes? \\
Können Schutz, Nutzung und Erholung überhaupt auf derselben Fläche stattfinden? \\
Sind die Konflikte die sich daraus ergeben forstpolitisch lösbar oder untragbar? \\
Mit welchen politischen Instrumenten geht man vor und wie eignen sich diese?

Alle diese Fragen gilt es im nachfolgenden Text zu beantworten. 
Meiner Meinung nach kann eine multifunktionale Waldwirtschaft ein erreichbares Ziel sein
und dies werde ich am Beispiel der Mittelwaldwirtschaft im Mooswald begründen. 


\section*{Diskussion}
\subsection*{Multifunktionale Waldwirtschaft}
Vorerst ist der Begriff der multifunktionalen Waldwirtschaft zu erklären. Maßgeblich ist
hier die Lehre der verschiedenen Leistungen des Waldes die unter den Begriffen Schutz-, Nutz-
und Erholungsfunktion des Waldes zusammengefasst werden. Daraus ergibt sich eine Betrachtungsweise
die über die Holzproduktion hinaus geht und in der Wechselbeziehungen zwischen Wald und Gesellschaft
erfasst werden. Forstpolitische Instrumente wie Gesetze, Anreize, Informationen und Strukturveränderung
werden zum Ausgleich zwischen den drei Waldfunktionen eingesetzt. 

Die Schutz- und Erholungsfunktion spielen erst seit den 1960er Jahren eine zunehmend wichtigere Rolle.
Bevor die Urproduktion durch Land- und Forstwirtschaft durch Industrie und Dienstleistungen immer weiter
abgelöst wurde, stand die Nutzfunktion im Vordergrund. Dies äußerte sich in Form des Bedarfsdeckungs-
und Versorgungsprinzipes (1200-1780) aber auch im erwerbswirtschaftlichen Prinzip (1780-1960).

\subsection*{Wiederaufnahme des Mittelwaldbetriebs im Opfinger Mooswald}
Die Mittelwaldwirtschaft ist eine Kombination aus einer kurzumtriebigen Niederwaldschicht und einem Überhaltbetrieb
mit ausgewählten Laubhölzern. In der Vergangenheit dienten diese Überhälter, meist Eichen, der landwirtschaftlichen
Nutzung, insbesondere der Schweinemast und später der Starkholznutzung. Während die Niederwaldschicht überwiegend
als Brennholz verwendet wurde. Diese forstliche Betriebsart führte zu einer engen Bindung vieler wärmeliebender 
Tierarten an diese Flächen. 

Seit ungefähr 100 Jahren wird im Freiburger Mooswald keine Mittelwaldwirtschaft mehr betrieben. Trotzdem 
sind die Überreste dieser historischen Nutzungsform in weiten Teilen des Mooswaldes zu sehen, obwohl die
Umwandlung in Hochwaldbetrieb seit über 70 Jahren stattfindet.

Da der Erhalt der typischen Lebensgemeinschaften eines Eichen-Hainbuchen-Mittelwaldes, wie er im Freiburger 
Mooswald zu finden ist, als Ziel im Vordergrund stand wurde 2002 im Opfinger Mooswald ein Pilotprojekt
zur Wiederaufnahme des Mittelwaldbetriebes gestartet. Auf einer Fläche von 24,6 ha orientierte sich der Forstbetrieb
bei der Rückführung in einen Mittelwald an der historischen Bewirtschaftung. Er bezog sich auf das
Forsteinrichtungswerk des Gemeindewaldes Opfingen von 1845, sowie den Beschreibungen von Huetlin (1874),
Hamm (1900) und Brandl (1970).\footnote{Coch, Müller-Bauernfeind 2002}

Hamm (1900) definierte den Mittelwald über die Umtriebszeit. Nur wenn das Oberholz ein mehrfaches an Lebensalter
des Unterholzes erzielte konnte von einem Mittelwald gesprochen werden. Auch zeigte er eine Reihe von
Fehlern auf, die seiner Meinung nach in der Bewirtschaftung stattfanden. Beispielsweise sprach Hamm von einer
zu langen Umtriebszeit der Niederwaldschicht oder von einer fehlerhaften Wahl der Baumarten. Außerdem
passte er die Bewirtschaftungsweise an den jeweiligen Standort an. Der sogenannte ''Staffelstand`` wies jeder
Gehölzschicht einen ihr zustehenden Kronenraum zu.\footnote{Hamm 1900}
Bei der Rückführung in der Abteilung 11 ``Obermoos`` wurde diese Kritik und Empfehlungen mit in der Planung beachtet.

Die Wiedereinführung des Mittelwaldhiebes führte zu einer Abweichung des Forsteinrichtungsplanes.
Ursprünglich war eine Entnahme von 20 fm/ha Holz im Jahrzehnt geplant. Der Holzvorrat belief sich auf 302 fm/ha. 
Durch die Umwandlung wurde die Abteilung 11 in 13 Flächen von je 1,8 ha Größe eingeteilt, in denen alle zwei 
Jahre ein Hieb durchzuführen ist, mit dem Ziel einen oberholzreichen Mittelwald mit etwa 120 fm/ha Vorrat in den
Überhältern zu erhalten. Die Hiebsführung orientierte sich maßgeblich an Huetlin (1847).\footnote{Coch, Müller-Bauernfeind 2002} Dieser beschrieb, dass
das Oberholz zwei bis vier Umtriebszeiten des Unterholzes über stand, nur die Eiche sollte mindestens sechs lang 
stehen.\footnote{Huetlin 1874} Wenn also alle 26 Jahre die Niederwaldschicht geerntet wurde, so blieb beispielsweise die Eiche im Überhalt 
156 Jahre lang stehen. Die Baumarten, die im Opfinger Mooswald gefördert werden sollen, sind Eichen und Eschen
als Starkholz und Hainbuchen, Haseln und Roterlen als Schwachholz. Je nach Bodenfeuchte stand die eine oder andere
Art im Vordergrund. Stieleichen bei trockeneren Gebieten und Eschen in feuchteren Senken.\footnote{Coch, Müller-Bauernfeind 2002}

Mit diesen Maßnahmen soll der Hochwald über einen längeren Zeitraum hinweg in einen Mittelwald zurückgeführt werden.


\subsection*{Diskussion: Waldwirtschaft - Angewandter Naturschutz oder Scheinharmonie?}
Die sogenannte ''Kielwassertheorie`` geht davon aus, dass durch die naturnahe Waldwirtschaft
Schutz und Erholung gewährleistet wird. Ist das ein Trugschluss? Kann die Forstwirtschaft 
angewandter Naturschutz sein oder besteht nur eine Scheinharmonie, welche durch die Presse 
schön geredet wird?

Es steht außer Frage, dass der Naturschutz einen Mehraufwand einen Minderertrag auf den
geschützten Flächen für die Forstwirtschaft bedeutet. Auch bei der Wiederaufnahme der 
Mittelwaldwirtschaft stehen diese beiden Punkte aus ökonomischer Sicht im Vordergrund. 
Heutzutage ist es nun einmal so, dass Stammholz wesentlich mehr Geld einbringt, als Brennholz.
Durch die Rückführung in einen Mittelwald, verzichtet die Forstwirtschaft auf einen
großen Teil der Stammholznutzung und hat durch den geminderten Ertrag Geldeinbußen. Aus dem
Niederwald erhalten sie nur Brenn- und Industrieholz mit denen die Verluste nicht wett gemacht 
werden können. 

Der Mehraufwand entsteht durch die arbeits- und kostenintensive Pflege, die der Zustandserhalt oder auch
die Rückführung zum Mittelwald mit sich bringt. Die Kosten für Arbeitsstunden werden nicht durch
den Ertrag gedeckt, was zu einem weiteren Geldverlust führt. Nun fordert der Mittelwald für seinen
Erhalt regelmäßige Pflegemaßnahmen. Ist diese finanzielle Belastung aus wirtschaftlicher Sicht überhaupt tragbar?
Es steht also soweit fest, dass ein Mittelwald aus wirtschaftlicher Sicht nicht 
mit einer Hochwaldnutzung konkurrieren kann.

Zudem wird von der Naturschutzseite vermehrt gefordert Prozessschutz zu betreiben. Somit ist eine
Rückführung und der strenge Zustandserhalt, also ein statischer Schutz, eines Gebietes nicht 
zwangsläufig gewünscht. Sie entspricht nicht einer natürlichen Waldentwicklung und könnte somit 
seine Gegner haben.

Es entstehen zwangsläufig Konflikte zwischen Schutz und Nutzung. Sollten diese zwei Bereiche also
Grundsätzlich streng voneinander getrennt werden, wie z.B. in einem Bannwald?

Meiner Überzeugung nach nicht. Denn Konflikte bieten immer die Möglichkeit einer positiven 
Weiterentwicklung. Natürlich kommt es immer wieder zu Problemen und Meinungsverschiedenheiten,
diese Fördern jedoch neue Lösungswege und gemeinsame Strategien.

Wie schon erwähnt, haben sich viele gerade wärmeliebende Arten an den strukturreichen Lebensraum Mittelwald angepasst.
Dies betrifft besonders geschützte Insekten, Vögel und Fledermäuse, welche somit auch als Kulturfolger
bezeichnet werden. Ohne den Erhalt der Strukturen eines Mittelwaldes, verlieren diese Tiere, aber auch
Pflanzen, ihre Lebensgrundlage. Sie würden aus unserem Mooswald verschwinden. Sprich, ohne Pflege kein
Schutz. Deshalb wird hier aus naturschutzfachlicher Sicht der integrierte Schutz und Zustandserhalt 
unterstützt, da trotz mäßig schwerer Durchführbarkeit das Aufwertungspotential sehr hoch ist. 
Zudem zählt die Fläche zum Landschaftsschutz- und FFH-Gebiet, sodass durch die Wiedereinführung der 
Mittelwaldbewirtschaftung der Biotopverbund zum Erhalt und Förderung seltener Arten begünstigt wird.

Dieses Beispiel zeigt gleichzeitig auch, dass nicht jeder Eingriff des Menschen 
in die Natur schlechtes mit sich bringt. Durch die menschliche Nutzung können demnach auch wunderbare
neue Dinge entstehen, deren Schutz sich lohnt.

Die finanziellen Einbußen wurden außerdem von vorne herein eingeschränkt. Die Fläche auf der die historische
forstliche Betriebsart wieder eingeführt wurde, war von vorne herein nicht sonderlich ertragreich.
Zusätzlich konnte der Holzeinschlag weitestgehend kostenneutral vonstatten gehen, da dieser zu einem großen Teil
durch Brennholz-Selbstwerber geschah. Somit wurde sogar ein ansehnlicher erntekostenfreier Ertrag eingenommen.
Der Pflegeaufwand kann so ohne nennenswerte Kosten von statten gehen. Dies entspricht auch der ''Lokalen Agenda 21``,
welche die regionale Holzvermarktung unterstützt.\citefooturldate{http://www.freiburg.de/servlet/PB/menu/1148519/index.html}{03.12.10}
Zusätzlich sollte auch nicht in
den Hintergrund rücken, dass Holz als Brennstoff wieder zunehmend wichtiger wird. Obwohl es heute viele
Ersatzstoffe für die Energieproduktion gibt, gewinnt Holz als regenerativer Rohstoff wieder an Bedeutung.
Vielleicht stellt der Mittelwald die zunünftige Waldbewirtschaftung dar, in der sowohl Brenn- als auch Starkholz
gewonnen wird ohne Verluste zu machen. Dementsprechend wird interessant werden, wie sich der Holzmarkt
in dieser Richtung entwickelt. 

Zudem existieren mittlerweile staatliche Förderprogramme zur Verwendung von Holz als
Energielieferant. Beispielsweise das ''Förderprogramm EnergieHolz Baden-Württemberg`` vom Ministerium Ländlicher
Raum oder aber auch das Bundesförderprogramm ''Maßnhamen zur Nutzung erneuerbarer Energien``. Zwar betreffen
diese finanziellen Anreize die Entwicklung von Heizanlagen und die Anschaffung für den Privatgebrauch, 
jedoch wirken sie sich auch nachhaltig auf die Forstwirtschaft aus.\footnote{Biotopverbundkonzept 2003}

In diesem Beispiel konnten, mit einfachen Vorgängen, die schwerwiegendsten Probleme gelöst, zum Teil sogar 
von vorne herein vermieden werden. Die in diesem Teil vorgestellten politischen Instrumente, können zwar 
einzeln betrachtet, aber erst in ihrem Zusammenspiel mehr über ihre Wirkung ausgesagt werden.
Auf der einen Seite stehen die gesetzlichen Vorgaben, die z.B. im Naturschutzgesetz und im Bundeswaldgesetz
zu finden sind. Sie bestehen aus Richtlinien, Geboten und Verboten die hier insbesondere für die 
Schutzgebiete beachtet werden müssen. Gerade diese Gesetzesvorgaben, lassen ökonomische Konflikte entstehen,
die aber auf der Anderen durch staatliche Förderprogramme als finanzielle Anreizsysteme, ausgeglichen werden. 
Sie führen zu Strukturveränderungen die sich förderlich auf z.B. den Absatz von Energieholz auswirken. 
Miteinander verbunden sind sie durch Kommunikation,
in der die Informationen weitergegeben werden. Der Vollständigkeit wegen, ist hier noch anzumerken, dass 
nicht alle der angewandten Instrumente dargestellt wurden.

Die Wiederbelebung des Mittelwaldbetriebes, hat zwar Konflikte
zwischen Ökonomie und Ökologie hervorgebracht, die aber mittels forstpolitischer Instrumente ausgeglichen
werden konnten. Demnach bin ich der Meinung, dass Spannungen zwischen Ökonomie und Ökologie über forstpolitische 
Methoden durchaus lösbar sind. 


\subsection*{Diskussion: Waldmensch oder Menschenwald?}
Dieser Teil der Diskussion beschäftigt sich nun mit der Zusammenwirkung der Waldfunktionen
Schutz und Erholung. Der stadtnahe Wald wird von vielen Menschen als Ort der Erholung angesehen,
den sie in ihrer Freizeit aufsuchen. Ist der Mensch nun eher als Störfaktor im Wald zu betrachten,
oder ist er als Teil der Natur zu sehen? \\

\noindent \textit{''Und Gott segnete sie und sprach zu ihnen: Seid fruchtbar und mehret euch und füllet die Erde und 
machet sie euch untertan und herrschet über die Fische im Meer und über die Vögel unter dem 
Himmel und über das Vieh und über alles Getier, das auf Erden kriecht.``} (1.Mose, 28) \\

\noindent Der Bibel zufolge steht der Mensch über allen anderen Wesen, ist also Teil der Schöpfung, aber
herrscht über die Welt. Auch wenn wir uns gerne als ''Waldmenschen`` bezeichnen würden, so lässt sich doch
eher die ''Untertanmachung`` belegen. 

Zu jeder tages- und nachtzeit ist der Wald für uns zugänglich um uns von den städtischen Strapazen zu
lösen. Morgens um fünf geht der erste Jogger in den Wald um die Einsamkeit im Morgengrauen zu genießen.
Dann kommen auch schon die Hundeausführer, Menschen die ihren täglichen
Nachmittagsrundgang vollführen, Familienausflüge, Wanderungen, baden im Baggersee, abends dann noch grillen
und und und .... 

Diese hohe Besucherstrom, wird tagtäglich vom Wald aufgenommen und hinterlässt Spuren. Unachtsam wird
Müll liegen gelassen, besonders abenteuerlustige gehen von Wegen ab um das pure wilde Naturerlebnis genießen
zu können, zerstören Pflanzen und verscheuchen Tiere.

Der Wald wird als etwas selbstverständliches hingenommen, er ist da und frei für alle Menschen. 
Wie steht es mit den Rechten für Tiere und Pflanzen? Heutzutage ist es etwas besonderes für
die meisten geworden tagsüber ein Reh oder anderes Wildtier zu sehen. Wieso? Sind Rehe nicht eigentlich
tagaktiv? Ja. Das sie heute eher im Morgengrauen oder im Abendrot zu sehen sind, zeigt eine Anpassung
an die ständige Aktivität im Wald. 
Jeder Naturgenießer und -liebhaber, würde schockiert über die Feststellung sein, dass sein Aufenthalt im
Wald ein Störfaktor darstellt.

Kann trotz des oft rücksichtslosen Verhaltens der Waldbesucher Schutz im Wald durchgeführt werden? Oder 
bedeutet Wiederbelebung der Mittelwaldwirtschaft, dass anstatt Schweinemast, Besucher die Fläche überbeanspruchen?
Auch hier steht also die Frage im Raum ob Schutz und Erholung flächenmäßig voneinander getrennt werden 
sollten. 

\subsection*{Diskussion: Holznutzung - Wald als Opfer oder im Einklang mit der Gesellschaft?}

\section*{Fazit und Ausblick}

\end{document}

