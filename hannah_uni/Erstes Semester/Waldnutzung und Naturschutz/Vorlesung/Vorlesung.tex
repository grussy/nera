\documentclass[12pt]{article}
\usepackage[paper=a4paper,left=30mm,right=30mm,top=35mm,bottom =35mm]{geometry}
\usepackage[utf8]{inputenc}
\usepackage[T1]{fontenc}
\usepackage{stmaryrd}
\usepackage{setspace}
\usepackage{mathrsfs}
\usepackage[ngerman]{babel}
\usepackage{amssymb}
\usepackage{amsmath}
\usepackage{fancyhdr}
\usepackage[dvips,unicode,colorlinks,linkcolor=black]{hyperref} 
\usepackage{graphicx}
\usepackage{float}

\begin{document}
\begin{titlepage}
\author{Hannah} 
\title{Vorlesung} 
\maketitle
\newpage
\tableofcontents
\end{titlepage} 


\section{Wissenschaftliches Arbeiten - Gero Becker 18.10.10}

\subsection*{Vorgehensweise}
\begin{itemize}
  \item Frage- und Problemstellung erfassen und definieren
  \item Ziele setzen
  \item Arbeitshypothese formulieren
  \item Material und Methoden wählen
\end{itemize}

\subsection*{Aufbau}
\begin{itemize}
 \item Einleitung
\begin{itemize}
 \item Problemstellung
\item Relevanz und Rechtfertigung um der Fragestellung nachzugehen
\item Forschungsstand: Welche Arbeiten gibt es schon darüber
\item Aus dem Forschungsstand Forschungsbedarf ableiten
\item Ziele aus den Arbeitsschritten ableiten
\begin{itemize}
 \item Arbeit und Material auf die Ziele auslegen
\end{itemize}
\end{itemize}
\item Material und Methoden
\begin{itemize}
 \item Welcher Forschungsansatz ist geeignet um die Arbeitshypothese zu formulieren
\item Welche Methoden stehen zur Verfügung?
\item Welches Material wird für die Beantwortung der Fragestellung benötigt?
\end{itemize}
\item Ergebnisse und Diskussion
\begin{itemize}
 \item Erkenntnisse mit anderen Forschungen vergleichen und bewerten.
\end{itemize}
\item Ausblick
\begin{itemize}
 \item Welche neuen Forschungsfragen ergeben sich aus der Arbeit?
\item Welche Arbeitsfelder sollte man noch einmal genauer untersuchen?
\end{itemize}
\end{itemize}

\subsection*{Beispiel Kahlschlagsfläche}
\begin{itemize}
 \item Problemstellung: Kahlschlagsfläche
 \item Problemdefinition: Flächige Befahrung (Bodenverdichtung); Erosion; Verjüngung; Ästhetik
\end{itemize}

\subsection*{Forschungsfragen und wissensbasierte Lösungen für die Forst- und Holzkette}
\subsubsection*{Beispiel Feinerschließung}
\begin{itemize}
 \item Walderschließung: Entwicklung und Praxiserprobung einer Testreihe zur Lokalisierung bodenmechanisch vorbelasteter Bodenareale
       von Forststandorten und die Einbeziehung dieser Fahrlinien in zukünftige Konzepte der Waldpflege und Holzernte.
\begin{itemize}
 \item Feinerschließungsnetz ist: Arbeitsort für Maschinen, Wirtschaftlichkeit, Waldbau, Bodenschutz und Bestandespfleglichkeit
 \item Maschinen auf dem Waldboden führen zu Bodenverdichtung, Wurzelschäden oder Schäden an Bäumen
 \item Diese Schäden können durch Maßnahmen verringert werden: Befahrung extrem trockener oder durchgefrorener Böden.
 \item Trotzdem besteht ein Verletzungsrisiko und man sollte deshalb diese Schäden auf bestimmte Flächen konzentrieren (Rückegassen).
 \item Diese Flächen sollten auch nicht ständig wechseln, sondern immer die gleichen Flächen betreffen
\end{itemize}
\item Problem: Risiko Waldboden
\item Ziel: Rückegassen konzentrieren; alte Befahrungslinien erkennen (Methodenvergleich); Integration identifizierter Befahrungslinien in
      zukunftsgerechte Erschließungssysteme unter standörtlichen und ökonomischen Gesichtspunkten. Erstellung eines Permanenten Feinerschließungsnetzes.
\item Fragestellung: Wie erkenne ich auf den vorgegebenen Linien zu bleiben?
                     Beurteilung Alternativen zur Erkennung alter Befahrungslinien (Lokalisierung im Gelände).
\item Relevanz: Lösung für das Problem auch zu gewährleisten, müssen die Wege wieder gefunden werden. (Um Schäden zu vermeiden)
\item Methoden: Witterungsbeständige Markierungen, Luftbilder, alte Nutzungskarten/ Pläne, Bodenvegetation (Verdichtungszeiger),
      Bodenproben nehmen, Laserscan (Fahrspuren erkennen anhand Wellenlängenresonanz; Waldboden optisch abtasten), Interviews mit
      Waldarbeitern, Förstern etc., Ortsbegang GPS
\item Forschungsstand klären: 
\begin{itemize}
 \item Mit dem Problem suchen: Fachlicher Zugang
 \item Methoden wählen und Problem darüber suchen: methodischer Ansatz
 \item Aufgrund der Literatur Methoden abschätzen
\end{itemize}
\item Forschungsansatz:
\begin{itemize}
 \item Vergleichsfläche von der man die genaue Lage der Wege kennt um die Methoden zu testen.
 \item Methoden nach Aufwand sortieren und bewerten
 \item Über die Vergleichsfläche kann man Methoden verifizieren und falsifizieren
\end{itemize}
\item Ergebnisse und Diskussion
\begin{itemize}
 \item Man wendet die Methoden an und legt diese übereinander
 \item Durch den Vergleich der verschiedenen Verfahren erhält man die Genauigkeit der Messmethode
 \item Bewertungskriterium: Genauigkeit und Vollständigkeit 
 \item Methode an unterschiedlichen Beständen testen:
\begin{itemize}
 \item Alt/Jung; Ebene/Hang; Laubholz/Nadelholz => Keine frischen Erntemaßnahmen und mit Testflächen
 \item Um unterschiedliche Methodegenauigkeiten festzustellen
 \item Für statistische Genauigkeit Wiederholungen 
\end{itemize}
 \item Zuverlässigkeit der Methode als Ergebnis und des Aufwandes
 \item Bewertung der Ergebnisse für Praktikabilität, Vernünftigkeit, Kosten, Zeitaufwand etc. 
 \item Erkennungsmöglichkeiten alter Befahrungslinien durch GPS-Aufnahme am genauesten
\end{itemize}
\item Ausblick: Optimierung/Umstrukturierung und Neuerschließung der Waldwege. (Integration alter Befahrungslinien.
\end{itemize}

\subsection*{Integrierende Optimierungsansätze für eine nachhaltige Energieholzversorgung}
\subsubsection*{Beispiel Bioenergie}


\end{document}
