\documentclass[12pt]{article}
\usepackage[utf8x]{inputenc}
\usepackage[paper=a4paper,left=30mm,right=30mm,top=35mm,bottom =35mm]{geometry}
\usepackage{stmaryrd}
\usepackage{setspace}
\usepackage{mathrsfs}
\usepackage[ngerman]{babel}
\usepackage{amssymb}
\usepackage{amsmath}
\usepackage{fancyhdr}
\usepackage[dvips,unicode,colorlinks,linkcolor=black]{hyperref} 
\usepackage{graphicx}
\usepackage{float}


\usepackage[multiple,marginal,ragged,hang]{footmisc}
\usepackage{cite} 
\usepackage{hyperref}  
\usepackage[babel,german=quotes]{csquotes}
\usepackage[T1]{fontenc}
\usepackage{multibib}


\newcommand{\citefooturl}[1]{\footnote{\url{#1}}}
\newcommand{\citefooturldate}[2]{\footnote{\url{#1} [#2]}}

%opening
\begin{document}

\clubpenalty = 10000
\widowpenalty = 10000 
\displaywidowpenalty = 10000

\onehalfspacing 


\begin{titlepage}

\title{Stadterweiterung – Verdichtung oder Ausdehnung? \\}
\author{\\ \\ \\ \\ vorgelegt von \\ \\
\begin{tabular}{ll}
 Carolin Creyaufmüller & Matr.-Nr. 2304883 \\
 Virginia Lorek & Matr.-Nr. 2304883 \\ 
 Sandra Neuwersch & Matr.-Nr. 3160841  \\ 
 Hannah Sharaf & Matr.-Nr. 2304883 \\ 
 Sofie Vallée & Matr.-Nr. 2304883 \\ \\
\end{tabular}}

\date{am \\ 09.11.2010}
\maketitle
\vfill {\noindent Institut für Forstbenutzung und Arbeitswissenschaften und Institut für Landespflege an der Albert-Ludwigs-Universität Freiburg. \\ Verantwortliche Dozenten: Herr Prof. Dr. Dr. h.c. Gero Becker, Herr Prof. Werner Konold, Dr. Leif Nutto, Dr. Harald Schaich.}
\thispagestyle{empty}
\newpage

\end{titlepage} 
\tableofcontents
\thispagestyle{empty}
\newpage

\section{Hypothetische Problemdefinition}
In den nächsten zwanzig Jahren, ist in Freiburg ein Stadtwachstum von fünf Prozent geplant. 
Für Neubebauungen stehen zwei Optionen zur Wahl. Entweder eine Verdichtung im Stadtkern 
oder eine Erschließung von Bauflächen in Waldrandbezirken. Um Vor- und Nachteile 
darzustellen bedarf es umfangreicher Studien. 
Das Hauptaugenmerk liegt auf vier Forschungsbereichen, welche die ökologischen und sozialen 
Interessen aufgreifen.

\section{Relevanz und Rechtfertigung}
Freiburg trägt den Titel „Grüne Stadt“ und dieser soll auch erhalten bleiben. 
Dementsprechend ist eine Forschung bei einem solchen Bauvorhaben auf ökologischer 
Basis von Nöten. Hinzu kommt, dass die Lebensqualität der Einwohner eine wichtige 
Rolle spielt und somit die soziale Komponente mit einbezogen werden muss. Nicht 
außer Acht zu lassen sind ebenso die ökonomischen Konsequenzen, die mit einer 
Stadterweiterung einhergehen.

\section{Ziele}
Ziel dieser Arbeit ist es die geeignetste ökologische, ökonomische und soziale Lösung zu 
ermitteln und daraus ein Empfehlungskonzept als Planungshilfe zu erstellen. 
Die aufgestellten Hypothesen werden im Laufe der Studie verifiziert oder falsifiziert.

\section{Hypothesen, Wissensstand und mögliche \\ Methoden}
\subsection{Wald und Landschaft}

\textbf{Hypothese:}	\textit{„Durch waldnahes Wohnen erhöhen sich die Kosten und der Aufwand für 
                 Verkehrssicherungs- und Holzerntemaßnahmen“} \\

\noindent Die Erschließung neuer Wohngebiete an den Waldrändern Freiburgs, 
erhöht die Verkehrsicherungspflicht für die Forstrevierleiter und Waldbesitzer. 
Dies gilt ebenso für neuangelegte öffentliche Straßen entlang der Wälder.
 
Die zuständigen Forstrevierleiter müssen regelmäßige Kontrollen an den Waldrändern durchführen und 
die Ergebnisse in einem Maßnahmenprotokoll 
notieren. In diesem Maßnahmenprotokoll muss im Falle eines Schadenfalls nachzuweisen sein,
dass während des Prüfganges keine erkennbaren Gefahren an den Bäumen zu finden waren. Neben der 
regelmäßigen Überprüfung der Waldränder auf Gefahrenquellen, müssen beispielsweise umsturzgefährdete 
Bäume beseitigt werden, wenn dadurch die Sicherheit von Gebäuden am Waldrand, öffentlicher 
Straßen und Erholungseinrichtungen wie z.B. Waldspielplätze, Sitzbänke oder Grillstellen gefährdet ist. 
\citefooturldate{http://www.lw-heute.de/index.php?redid=17800}{05.11.2010}
Eine weitere Problematik, die aus den waldnahen Wohngebieten resultiert, ist die steigende Anzahl an 
Erholung suchenden Waldbesucher. Denn durch den steigenden Besucherandrang im Wald muss besondere 
Sorgfalt auf die Kontrolle von stehendem Totholz entlang der Waldwege gelegt werden. 
Dieser erhöhte Bedarf an Kontrollmaßnahmen führt zu steigenden Personalkosten für die Stadt Freiburg 
im Bereich des Stadtwaldes und für den Landesbetrieb Forst BW im Bereich des Staatswaldes.


Um herauszufinden, wie hoch der Mehraufwand an Verkehrsicherungspflicht im Falle einer Ausdehnung der 
Wohngebiete am Waldrand in Realität ist, sollten Expertengespräche mit betroffenen Forstrevierleitern 
geführt werden. Ebenso empfiehlt sich die Methode, die Maßnahmenprotokolle für die Verkehrssicherungspflicht 
der Revierleiter zu überprüfen, wie hoch der Zeitbedarf und die daraus ableitbaren Kosten tatsächlich sind.


Auch die Holzerntekosten werden mit vermehrten Vorkommen von Wohngebieten an Waldrändern steigen. Es müssten
in bestimmten Fällen besondere Fälltechniken mit einem höheren Sicherheitsstandard angewandt werden, 
beispielsweise „seilunterstütztes Fällen“ und darüber hinaus müssen durch die steigende Anzahl an Waldbesuchern 
die Absperrmaßnahmen während den Holzerntemaßnahmen intensiviert werden, insbesondere müssten Streckenposten 
gestellt werden.


Expertengespräch mit der Unteren Forstbehörde und den betreffenden Forstrevierleitern, welche besonderen 
Holzerntetechniken in Beständen an angrenzenden Wohngebieten angewendet werden. Prüfung des erntekostenfreien 
Erlös und der Holzerntekosten in den Wirtschaftsbüchern der Forstreviere mit entsprechenden Beständen und 
Vergleich der Holzerntekosten aus Vergleichsbeständen ohne angrenzende Wohngebiete. \\ \\


\noindent\textbf{Hypothese:}	\textit{„Mit Zunahme der Landschaftszerschneidung verringert sich die Biodiversität.“} \\

\noindent Unter Landschaftszerschneidung wird die räumliche Trennung von Landschaftselementen und/oder gewachsenen 
ökologischen Zusammenhängen in der Fläche verstanden.
Durch den Bau von Wohnsiedlungen entlang der Waldränder werden bestehende Lebensräume zerschnitten. 
Sie stellen für viele Tier- und Pflanzenarten ein Hindernis dar und zerkleinern, zerteilen und isolieren 
deren Lebensräume. Dies hat zur Folge, dass die Artenvielfalt abnimmt. 
\citefooturldate{http://de.wikipedia.org/wiki/Landschaftszerschneidung}{05.11.2010}

Um darstellen zu können, dass durch den Bau von Häusern entlang des Waldrandes die Landschaftszerschneidung 
zunimmt und dadurch die Biodiversität sinkt, bietet sich die Bestimmung des Zerschneidungsgrades mit Hilfe 
der effektiven Maschenweite an. 

\begin{quote}
„Die Effektive Maschenweite ist ein Indikator, der die Flächengrößen der unzerschnittenen Räume in das 
Verhältnis zur Gesamtfläche setzt und so einen relativen Wert für den Grad der Zerschneidung angibt. Je 
größer der Ergebniswert ist, desto geringer ist die Landschaft zerschnitten.“
\citefooturldate{http://www.tu-dresden.de/ioer/statisch/langzeitmonitoring_uzf/analyse_maschenw.html}{05.11.2010}
\end{quote}


Die effektive Maschenweite lässt sich mit folgender Formel ermitteln:
\citefooturldate{http://www.lubw.baden-wuerttemberg.de/servlet/is/20280/meff_tool.pdf?command=downloadContent&filename=meff_tool.pdf}{05.11.2010}
\begin{align*}
 m_{eff} \quad = \quad F_g ~ \cdot ~ \sum\limits_{i=1}^{n}\left(\frac{F_i}{F_g}\right)^2
\end{align*}
mit $m_{eff}$ Zerschneidungsgrad, $F_g$ Gesamtfläche und $F_i$ Teilfläche.

\subsection{Tiere und Pflanzen}

\textbf{Hypothese:} \textit{„Durch die Lebensraumverkleinerung und neuen Nahrungsressourcen \\ drängen immer mehr Wildtiere in waldangrenzende 
            Stadtgebiete.“} \\ 

\noindent In den Stadtgebieten begegnen uns immer öfter Wildtiere. Aufgrund verkleinerter Lebensräume und der Zunahme 
des Nahrungsangebotes durch den städtischen Müll steigt die Anzahl gesichteter Tiere. Es stellt sich die Frage, ob 
in waldangrenzenden Gebieten mehr Wildtiere in Erscheinung treten, als im Stadtzentrum.

Als erster Schritt sollte eine Literaturrecherche aufzeigen, welche wissenschaftlichen Arbeiten sich bereits
mit dem Thema beschäftigt haben und welche Resultate erzielt wurden. 

Zusätzlich liefert die Methode der Befragung eigene Ergebnisse. Als 
Erhebungsmethode könnten persönliche und telefonische Interviews gewählt werden.
Anhand eines vorgefertigten Fragebogens, wäre eine Vergleichbarkeit der Ergebnisse durch Herausfilterung
der Kernaussage möglich. Doch auch spontane Fragen sollten Raum finden.
Die Erhebung müsste einerseits im Stadtzentrum und andererseits im waldnahen Raum stattfinden. 
Die wichtigsten Zielgruppen für die Interviews wären Experten des forstlichen und dem jagdlichen Bereiches.
So ließe sich auch eine Statistik über Beschwerden bezüglich der Schäden und Gefahren durch Wildtiere erstellen. 
Zusätzlich sollten die Sichtungen durch waldnahe und zentrale 
Anwohner aufgenommen und verglichen werden.

Falls eine wissenschaftliche Arbeit mit Habitatnutzungsanalyse zur Verfügung steht, könnte diese zu Rate gezogen werden.
Eine eigene Untersuchung würde den Rahmen einer Masterarbeit sprengen. 
Das Kerninteresse würde sich auf die Telemetrie von Tieren, Spurenkartierungen 
und Populationsermittlungen konzentrieren. Erhebungsmethoden wären z.B. Tierbesenderung, Fotofallen und Fang-Wiederfang. 
Damit ließen sich Raumnutzung und die Streifgebiete von Wohngebieten in der Innenstadt und 
in Waldnähe in Relation setzen. \\

\textbf{Hypothese:} \textit{„Die Biodiversität am Waldrand ist höher als in Stadtgrünflächen.“} \\

\noindent Strukturreiche Waldränder zählen nach dem Landeswaldgesetz § 30 a zu den geschützten Biotopen 
im Wald und sind wichtige Trittsteinbiotope im Rahmen der Vernetzung. \citefooturldate{http://www.fva-bw.de/publikationen/merkblatt/mb_48.pdf}{03.11.10} 
Deshalb sollten die in Frage kommenden Baugebiete untersucht werden, um die Schutzstellung vor dem Bau zu klären. 
Auch die Stadtgrünflächen müssen auf Schutzwürdigkeit geprüft werden. 

Vorab sollte eine Literaturrecherche über bereits bestehende Forschungsarbeiten zu diesem Thema stattfinden. 

Auf Versuchsflächen in den möglichen Bebauungsstandorten könnte anhand festgelegter Kriterien mit Hilfe von 
Punkt-, Linien- oder Rasterkartierung eine Vegetations-, Biotop- und Habitatanalyse durchgeführt werden. 
Ein Vergleich zwischen Stadtgrünflächen und Waldrandflächen gestaltet sich auf Pflanzen bezogen nicht einfach, 
da in der Stadt meist eine Bepflanzung
durch den Menschen stattfindet und diese unter ständiger Pflege steht. Trotzdem gibt es auch hier von Außen eingetragene 
seltene Wildpflanzen, die sich auf Extremstandorten ansiedeln. Tiere hingegen findet man auf beiden Flächen, sie lassen sich 
leichter miteinander vergleichen.
Diese Methode ermöglicht es geschützte Standorte und Rote Liste Arten auszumachen und Kriterien gegen eine Bebauung aufzuzeigen. 

\subsection{Naherholungsgebiet}

\textbf{Hypothese:} \textit{„Mit Zunahme der Verdichtung steigt die Aktivität in den Naherholungsgebieten  des Stadtwaldes.“} \\

\noindent Ein Naherholungsgebiet definiert sich als unbebautes Gebiet, das von Einwohnern aus Großstädten und Ballungsgebieten 
zur Erholung und Freizeitgestaltung aufgesucht wird. 
Derzeit gibt es zahlreiche Studien zur  Besuchererfassung in Naherholungsgebieten. Der Freistaat Thüringen erarbeitete 
beispielweise im Jahr 2006 eine Tagungsreihe zum Thema „Besuchermonitoring und ökonomische Effekte in Nationalen 
Naturlandschaften“. Es empfiehlt sich Werte aus Vergleichsstudien zur näheren Betrachtung heranzuziehen. Eine 
selbstständige Analyse der Bedingungen und Begebenheiten vor Ort bleibt jedoch für ein repräsentatives Ergebnis unabdingbar.
Literaturrecherchen haben ergeben, dass Besuchermonitoring in Naherholungsgebieten eines der wichtigsten Erfolgsinstrumente 
darstellt.\citefooturldate{http://mgross.hs-harz.de/Liste_Diplomarbeiten_ST.pdf}{04.11.10} In der Praxis erfolgt die 
Besuchererfassung in der Regel durch Zählungen. Diese können mit einer Vielzahl von 
Hilfsutensilien durchgeführt werden. Hierzu zählen insbesondere Lichtschranken und Vollzählungen. Eine Lichtschranke ist ein 
elektronisch-optisches System, das einen Lichtstrahl aussendet. Durch die Unterbrechung des Lichtstrahls erfolgt die 
Zählung. Die Vollzählung wird von mehreren Personen durchgeführt, die sich an stark frequentierten Stellen positionieren, 
um den Besucherverkehr zu dokumentieren. Beide Methoden führen zu einer Auflistung des Besucherverkehrs im jeweiligen 
Naherholungsgebiet. 
Eine weitere Methodik besteht in der Befragung der Stadtbevölkerung mit Hilfe eines erstellten Fragebogens. Dieser sollte 
so aufgebaut sein, dass ein Freizeitcluster erstellt werden 
kann.\citefooturldate{ttp://www.biosphaerenreservat-vessertal.de/dateien/6service/10ftagung/2006/taband06-screen.pdf}{04.11.10} 
Die Analyse des Fragebogens ergibt die 
Freizeitgewohnheiten der Befragten einerseits, den Ort an dem diese ausgeführt werden andererseits. Für eine optimale 
Einschätzung des Freizeitverhaltens bedarf es des Vergleiches der Stadtbevölkerung mit der ländlich lebenden Bevölkerung. 
Beide Methoden lassen Rückschlüsse auf die Aktivität in Naherholungsgebieten zu. 
\subsection{Klima}

\section{Ausblick und Diskussion}

\end{document}
