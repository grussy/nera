\documentclass[12pt]{article}
\usepackage[utf8x]{inputenc}
\usepackage[paper=a4paper,left=30mm,right=30mm,top=35mm,bottom =35mm]{geometry}
\usepackage{stmaryrd}
\usepackage{setspace}
\usepackage{mathrsfs}
\usepackage[ngerman]{babel}
\usepackage{amssymb}
\usepackage{amsmath}
\usepackage{fancyhdr}
\usepackage[dvips,unicode,colorlinks,linkcolor=black]{hyperref} 
\usepackage{graphicx}
\usepackage{float}


\usepackage[multiple,marginal,ragged,hang]{footmisc}
\usepackage{cite} 
\usepackage{hyperref}  
\usepackage[babel,german=quotes]{csquotes}
\usepackage[T1]{fontenc}
\usepackage{multibib}


\newcommand{\citefooturl}[1]{\footnote{\url{#1}}}
\newcommand{\citefooturldate}[2]{\footnote{\url{#1} [#2]}}

%opening
\begin{document}

\clubpenalty = 10000
\widowpenalty = 10000 
\displaywidowpenalty = 10000

\onehalfspacing 


\begin{titlepage}

\title{Stadterweiterung – Verdichtung oder Ausdehnung?}
\author{vorgelegt von \\ 
 Carolin Creyaufmüller \quad Matr.-Nr. 2304883 \\
 Virginia Lorek \quad Matr.-Nr. 2304883 \\ 
 Sandra Neuwersch \quad Matr.-Nr. 3160841  \\ 
 Hannah Sharaf \quad Matr.-Nr. 2304883 \\ 
 Sofie Vallée \quad Matr.-Nr. 2304883 \\ }
\date{09.11.2010}
\maketitle
\vfill {Verantwortliche Dozenten: Herr Prof. Dr. Dr. h.c. Gero Becker, Herr Prof. Werner Konold, Dr. Leif Nutto, Dr. Harald Schaich} 
\vfill {\noindent Institut für Forstbenutzung und Arbeitswissenschaften und Institut für Landespflege an der Albert-Ludwigs-Universität Freiburg}

\newpage
\tableofcontents
\end{titlepage} 


\section{Hypothetische Problemdefinition}
In den nächsten zwanzig Jahren, ist in Freiburg ein Wachstum von fünf Prozent geplant. 
Für Neubebauungen stehen zwei Optionen zur Wahl. Entweder eine Verdichtung im Stadtkern 
oder eine Erschließung von Bauflächen in Waldrandbezirken. Um Vor- und Nachteile 
darzustellen Bedarf es umfangreicher Studien. 
Hauptaugenmerk liegt auf vier Forschungsbereichen, die ökologische und soziale 
Interessen aufgreifen.

\section{Relevanz und Rechtfertigung}
Freiburg trägt den Titel „Grüne Stadt“ und dieser soll auch erhalten bleiben. 
Dementsprechend ist eine Forschung bei einem solchen Bauvorhaben auch auf ökologischer 
Basis von Nöten. Hinzu kommt, dass auch die Lebensqualität der Einwohner eine wichtige 
Rolle spielt und somit auch die soziale Komponente mit einbezogen werden muss. Nicht 
außer Acht zu lassen sind auch die ökonomischen Konsequenzen, die mit einer 
Stadterweiterung einhergehen.

\section{Ziele}
Ziel dieser Arbeit ist es die geeignetste ökologische, ökonomische und soziale Lösung zu 
erhalten und daraus eine Ableitung eines Empfehlungskonzeptes als Planungshilfe zu erzielen. 
Die aufgestellten Hypothesen werden im Laufe der Studie verifiziert oder falsifiziert.

\section{Hypothesen, Wissensstand und mögliche \\ Methoden}
\subsection{Wald und Landschaft}

\textbf{Hypothese:}	\textit{„Durch waldnahes Wohnen erhöhen sich die Kosten und der Aufwand für 
                 Verkehrssicherungs- und Holzerntemaßnahmen“} \\

Wenn immer mehr neue Wohngebiete entlang der Waldränder von Freiburg erschlossen werden, 
erhöht sich die Verkehrsicherungspflicht für die Forstrevierleiter und Waldbesitzer. 
Dasselbe gilt bei öffentlichen Straßen, die entlang der Wälder gebaut werden könnten
 
Die betroffenen Forstrevierleiter müssen regelmäßige Kontrollen an den Waldrändern durchführen. 
Die Ergebnisse dieser Kontrollegänge müssen vom Forstrevierleiter in einem Maßnahmenprotokoll 
notiert werden. In diesem Maßnahmenprotokoll muss im Falle eines Schadenfalls nachzuweisen sein,
dass während dem Prüfgang keine erkennbaren Gefahren an den Bäumen zu finden waren. Neben der 
regelmäßigen Überprüfung der Waldränder auf Gefahrenquellen müssen beispielsweise umsturzgefährdete 
Bäume beseitigt werden, wenn durch diese Bäume die Sicherheit von Gebäuden am Waldrand, öffentlicher 
Straßen und Erholungseinrichtungen wie z.B. Waldspielplätze, Sitzbänke oder Grillstellen gefährdet ist. 
\citefooturl{http://www.lw-heute.de/index.php?redid=17800 [05.11.2010]}
Eine weitere Problematik, die aus den waldnahen Wohngebieten resultiert, ist die steigende Anzahl an 
Erholung suchenden Waldbesucher. Denn durch den steigenden Besucherandrang im Wald muss besondere 
Sorgfalt auf die Kontrolle von stehendem Totholz entlang der Waldwege gelegt werden. 
Dieser erhöhte Bedarf an Kontrollmaßnahmen führt zu steigenden Personalkosten für die Stadt Freiburg 
im Bereich des Stadtwaldes und für den Landesbetrieb Forst BW im Bereich des Staatswaldes.


Um herauszufinden, wie hoch der Mehraufwand an Verkehrsicherungspflicht im Falle einer Ausdehnung der 
Wohngebiete am Waldrand in Realität ist, sollten Expertengespräche mit betroffenen Forstrevierleitern 
geführt werden. Ebenso empfiehlt sich die Methode, die Maßnahmenprotokolle für die Verkehrssicherungspflicht 
der Revierleiter zu überprüfen, wie hoch der Zeitbedarf und die daraus ableitbaren Kosten tatsächlich sind.


Auch die Holzerntekosten werden mit vermehrten Vorkommen von Wohngebieten an Waldrändern steigen. Es müssten
in bestimmten Fällen besondere Fälltechniken mit einem höheren Sicherheitsstandard angewandt werden, 
beispielsweise „seilunterstütztes Fällen“ und darüber hinaus müssen durch die steigende Anzahl an Waldbesuchern 
die Absperrmaßnahmen während den Holzerntemaßnahmen intensiviert werden, insbesondere müssten Streckenposten 
gestellt werden.


Expertengespräch mit der Unteren Forstbehörde und den betreffenden Forstrevierleitern, welche besonderen 
Holzerntetechniken in Beständen an angrenzenden Wohngebieten angewendet werden. Prüfung des erntekostenfreien 
Erlös und der Holzerntekosten in den Wirtschaftsbüchern der Forstreviere mit entsprechenden Beständen und 
Vergleich der Holzerntekosten aus Vergleichsbeständen ohne angrenzende Wohngebiete. \\ \\


\noindent\textbf{Hypothese:}	\textit{„Mit Zunahme der Landschaftszerschneidung verringert sich die Biodiversität.“} \\

Unter Landschaftszerschneidung wird die räumliche Trennung von Landschaftselementen und/oder gewachsenen 
ökologischen Zusammenhängen in der Fläche verstanden.
Durch den Bau von Wohnsiedlungen entlang der Waldränder werden bestehende Lebensräume zerschnitten. 
Sie stellen für viele Tier- und Pflanzenarten ein Hindernis dar und zerkleinern, zerteilen und isolieren 
deren Lebensräume. Dies hat zur Folge, dass die Artenvielfalt abnimmt. 
\citefooturldate{http://de.wikipedia.org/wiki/Landschaftszerschneidung}{05.11.2010}

Um darstellen zu können, dass durch den Bau von Häusern entlang des Waldrandes die Landschaftszerschneidung 
zunimmt und dadurch die Biodiversität sinkt, bietet sich die Bestimmung des Zerschneidungsgrades mit Hilfe 
der effektiven Maschenweite an. 

\begin{quote}
„Die Effektive Maschenweite ist ein Indikator, der die Flächengrößen der unzerschnittenen Räume in das 
Verhältnis zur Gesamtfläche setzt und so einen relativen Wert für den Grad der Zerschneidung angibt. Je 
größer der Ergebniswert ist, desto geringer ist die Landschaft zerschnitten.“
\citefooturl{http://www.tu-dresden.de/ioer/statisch/langzeitmonitoring_uzf/analyse_maschenw.html [05.11.2010]}
\end{quote}


Die effektive Maschenweite lässt sich mit folgender Formel ermitteln:
\citefooturl{http://www.lubw.baden-wuerttemberg.de/servlet/is/20280/meff_tool.pdf?command=downloadContent&filename=meff_tool.pdf [05.11.2010]}
\begin{align*}
 m_{eff} \quad = \quad F_g ~ \cdot ~ \sum\limits_{i=1}^{n}\left(\frac{F_i}{F_g}\right)^2
\end{align*}
mit $m_{eff}$ Zerschneidungsgrad, $F_g$ Gesamtfläche und $F_i$ Teilfläche.

\subsection{Tiere und Pflanzen}

\textbf{Hypothese:} \textit{„Durch die Lebensraumverkleinerung und neuen Nahrungsressourcen \\ drängen immer mehr Wildtiere in waldangrenzende 
            Stadtgebiete.“} \\ 

Wildtiere in der Stadt begegnen uns immer öfter. Durch die Verkleinerung der Lebensräume und der Zunahme 
des Nahrungsangebotes durch den städtischen Müll erhöht sich der Anteil gesichteter Tiere. Die Frage ist 
nun, ob der Anteil der Wildtiere in waldangrenzenden Gebieten höher ist, als im Stadtzentrum.

Vorerst wäre zu klären, welche wissenschaftlichen Arbeiten sich mit der Tieren in stadtnahen Gebieten und 
Tieren in stadtnahen Gebieten beschäftigen um den Stand des Wissens zu klären. 

Zusätzlich wäre die Befragung eine Methode um eigene Ergebnisse zu erzielen. Durch eine Zufallsstichprobe 
erhält man repräsentative  Informationen über \\ Meinungen, Wissen und Verhalten der befragten Personen. Als 
Erhebungsmethode kämen persönliche und telefonische Interviews zum Zuge. Die Erhebung müsste einerseits im 
Stadtzentrum und  andererseits im waldnahen Raum stattfinden. Die wichtigsten Personengruppen, die von der 
Umfrage betroffen wären, sollten aus dem forstlichen und dem jagdlichen Bereich stammen um eine Statistik 
über Beschwerden über Wildtiere zusammenzustellen. Dazu sollten die Sichtungen durch waldnahe und zentrale 
Anwohner aufgenommen und verglichen werden.

Desweiteren sollte eine Habitatnutzungsanalyse  zu Rate gezogen werden, insofern hierzu eine 
wissenschaftliche Arbeit besteht. Eine eigene Untersuchung würde den Rahmen einer Masterarbeit sprengen. 
Der grundsätzliche Aufbau dieses Forschungsteiles würde sich auf die Telemetrie von Tieren, Spurenkartierungen 
und Populationsermittlungen beziehen. Erhebungsmethoden wären z.B. Tierbesenderung, Fotofallen und Fang-Wiederfang. 
Die Raumnutzung und die Streifgebiete der \\ beobachteten Wildtiere können zwischen Wohngebiet in der Innenstadt und 
in waldnähe ins Verhältnis gesetzt werden. \\

\textbf{Hypothese:} \textit{„Die Biodiversität am Waldrand ist höher als in Stadtgrünflächen.“} \\

Strukturreiche Waldränder zählen nach dem Landeswaldgesetz § 30 a zu den geschützten Biotopen 
im Wald und sind wichtige Trittsteinbiotope im Rahmen der Vernetzung. \citefooturl{http://www.fva-bw.de/publikationen/merkblatt/mb_48.pdf} 
Deshalb sollten die in Frage kommenden Baugebiete untersucht werden, um die Schutzstellung vor dem Bau zu klären. 
Auch die Stadtgrünflächen müssen auf Schutzwürdigkeit untersucht werden. 

Vorerst sollte eine Literaturrecherche über bereits bestehende Forschungsarbeiten in diesem Gebiet stattfinden. 

Zusätzlich sollten Versuchsflächen in den möglichen Bebauungsstandorten gewählt werden und  mit Hilfe von 
Vegetations-, Biotop- Raster- und Habitatkartierung durchgeführt werden. So lassen sich \\ mögliche geschützte 
Standorte und Rote Liste Arten ausmachen, die gegen eine Bebauung sprechen könnten. 

\subsection{Naherholung}
\subsection{Klima}

\section{Ausblick und Diskussion}

\end{document}
